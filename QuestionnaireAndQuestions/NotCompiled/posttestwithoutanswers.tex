\documentclass{article}
\usepackage{enumitem}
\usepackage{amsmath, amssymb}
\usepackage[a4paper,margin=1in]{geometry}

\begin{document}

\section*{Domande post-test}

\begin{enumerate}[label=\textbf{Domanda \arabic*.}]

\item Qual è la differenza fondamentale tra un qubit e un bit classico?
\begin{enumerate}[label=\Alph*.]
\item Un qubit può rappresentare più di due stati alla volta
\item Un qubit può essere in una sovrapposizione di stati 0 e 1
\item Un qubit può memorizzare più informazioni di un bit classico quando misurato
\item Un qubit è più veloce di un bit classico nell'eseguire calcoli
\end{enumerate}
Risposta corretta: B

\item Come viene rappresentato un cbit \(|1\rangle\) in notazione vettoriale?
\begin{enumerate}[label=\Alph*.]
    \item \( \begin{pmatrix} 0 \\ 1 \end{pmatrix} \)
    \item \( \begin{pmatrix} 1 \\ 1 \end{pmatrix} \)
    \item \( \begin{pmatrix} 1 \\ 0 \end{pmatrix} \)
    \item \( \begin{pmatrix} 0 \\ 0 \end{pmatrix} \)
\end{enumerate}
Risposta corretta: A

\item Considerando il contesto delle operazioni quantistiche, quale delle seguenti trasformazioni su un qubit è considerata completamente reversibile?
    \begin{enumerate}[label=\Alph*.]
        \item Applicazione di una porta C-NOT
        \item Applicazione di una porta C-NOT seguita da una porta HADAMARD
        \item Applicazione di una porta bit-flip (negazione)
        \item Tutte le precedenti
    \end{enumerate}
    Risposte corrette: D

\item Qual è la dimensione del vettore di uno stato prodotto di \( n \) cbit?
\begin{enumerate}[label=\Alph*.]
    \item \( n \)
    \item \( 2^n \)
    \item \( n^2 \)
    \item \( 2n \)
\end{enumerate}
Risposta corretta: B

\item Qual è il principale vincolo che ha un qubit 
\begin{pmatrix}
    a \\ b
\end{pmatrix}?
\begin{enumerate}[label=\Alph*.]
\item \(a^2 + b^2 = 1\)
\item \(|a|^2 + |b|^2 = 1\)
\item \(a + b = 1\)
\item \(|a| + |b| = 1\)
\end{enumerate}
Risposta corretta: B

\item Considerando l'operatore CNOT nel quantum computing, quale delle seguenti affermazioni è vera?
\begin{enumerate}[label=\Alph*.]
\item Flippa il qubit target se e solo se il qubit di controllo è \(|0\rangle\).
\item Modifica il qubit di controllo basandosi sullo stato del qubit target.
\item Flippa il qubit target se e solo se il qubit di controllo è \(|1\rangle\), mentre il qubit di controllo rimane invariato.
\item Se il qubit di controllo è \(|1\rangle\), entrambi i qubit, di controllo e target, vengono flippati.
\end{enumerate}
Risposta corretta: C

\item Considerando un qubit il cui stato è rappresentato come
\begin{pmatrix}
    \sqrt{3}/2 \\ 1/2
    \end{pmatrix}, quale delle seguenti affermazioni descrive correttamente le probabilità di collasso del qubit quando viene misurato?
\begin{enumerate}[label=\Alph*.]
\item Ha una probabilità di \(3/4\) di collassare a \(|0\rangle\) e una probabilità di \(1/4\) di collassare a \(|1\rangle\).
\item Ha una probabilità di \(1/2\) di collassare a \(|0\rangle\) e una probabilità di \(\sqrt{3}/2\) di collassare a \(|1\rangle\).
\item Ha una probabilità di \(\sqrt{3}/2\) di collassare a \(|0\rangle\) e una probabilità di \(1/2\) di collassare a \(|1\rangle\).
\item Ha una probabilità di \(2/3\) di collassare a \(|0\rangle\) e una probabilità di \(1/3\) di collassare a \(|1\rangle\).
\end{enumerate}
Risposta corretta: A


\item Quale delle seguenti matrici rappresenta l'operatore CNOT?
\begin{enumerate}[label=\Alph*.]
    \item \(\begin{pmatrix} 1 & 0 & 0 & 0 \\ 0 & 1 & 0 & 0 \\ 0 & 0 & 0 & 1 \\ 0 & 0 & 1 & 0 \end{pmatrix}\)
    \item \(\begin{pmatrix} 0 & 1 & 0 & 0 \\ 1 & 0 & 0 & 0 \\ 0 & 0 & 1 & 0 \\ 0 & 0 & 0 & 1 \end{pmatrix}\)
    \item \(\begin{pmatrix} 1 & 0 & 0 & 0 \\ 0 & 0 & 0 & 1 \\ 0 & 0 & 1 & 0 \\ 0 & 1 & 0 & 0 \end{pmatrix}\)
    \item \(\begin{pmatrix} 0 & 1 & 0 & 0 \\ 0 & 0 & 0 & 1 \\ 1 & 0 & 0 & 0 \\ 0 & 0 & 1 & 0 \end{pmatrix}\)
\end{enumerate}
Risposta corretta: A 

\item Qual è l'effetto dell'operazione "Constant Zero" su un bit?
\begin{enumerate}[label=\Alph*.]
    \item Lo setta a 1
    \item Lo setta a 0
    \item Lo inverte
    \item Lo lascia inalterato
\end{enumerate}
Risposta corretta: B

\item Cos'è la "quantum supremacy"?
\begin{enumerate}[label=\Alph*.]
    \item La dimostrazione che i computer quantistici possono risolvere tutti i problemi più velocemente dei computer classici
    \item L'asserzione che la meccanica quantistica è superiore alla meccanica classica
    \item Il punto in cui un computer quantistico esegue un compito specifico in modo ineguagliabile e più velocemente da un computer classico
    \item La teoria secondo cui tutti i computer futuri saranno quantistici
\end{enumerate}
Risposta corretta: C



\item Dato lo stato \(|01\rangle\), quale delle seguenti rappresentazioni in forma di vettore colonna è corretta?
\begin{enumerate}[label=\Alph*.]
    \item \(\begin{pmatrix} 1 \\ 0 \\ 0 \\ 0 \end{pmatrix}\)
    \item \(\begin{pmatrix} 0 \\ 1 \\ 0 \\ 0 \end{pmatrix}\)
    \item \(\begin{pmatrix} 0 \\ 0 \\ 1 \\ 0 \end{pmatrix}\)
    \item \(\begin{pmatrix} 0 \\ 0 \\ 0 \\ 1 \end{pmatrix}\)
\end{enumerate}
Risposta corretta: B

\item Che cos'è il prodotto tensoriale tra due vettori?
\begin{enumerate}[label=\Alph*.]
    \item La somma dei vettori
    \item Il prodotto scalare dei vettori
    \item Un nuovo vettore ottenuto moltiplicando ogni elemento del primo vettore con ogni elemento del secondo
    \item La differenza tra i vettori
\end{enumerate}
Risposta corretta: C

\item Quale delle seguenti matrici rappresenta l'operatore Hadamard (H)?
\begin{enumerate}[label=\Alph*.]
    \item \(\frac{1}{\sqrt{2}} \begin{pmatrix} 1 & 1 \\ 1 & -1 \end{pmatrix}\)
    \item \(\frac{1}{2} \begin{pmatrix} 1 & -1 \\ 1 & 1 \end{pmatrix}\)
    \item \(\begin{pmatrix} 0 & 1 \\ 1 & 0 \end{pmatrix}\)
    \item \(\frac{1}{\sqrt{2}} \begin{pmatrix} 1 & -1 \\ -1 & 1 \end{pmatrix}\)
\end{enumerate}
Risposta corretta: A

\item Nel contesto del calcolo quantistico, quale porta è fondamentale come il NAND nel calcolo classico?
\begin{enumerate}[label=\Alph*.]
    \item Porta di Hadamard
    \item Porta di Pauli-X
    \item Porta CNOT
    \item Porta Toffoli
\end{enumerate}
Risposta corretta: C


\item Cosa significa quando un qubit è in uno stato di sovrapposizione?
\begin{enumerate}[label=\Alph*.]
    \item Il qubit è sia in stato \( |0\rangle \) che \( |1\rangle \)
    \item Il qubit è in uno stato indefinito
    \item Il qubit è in uno stato di errore
    \item Il qubit è in uno stato di entanglement
\end{enumerate}
Risposta corretta: A
\end{enumerate}


\end{document}