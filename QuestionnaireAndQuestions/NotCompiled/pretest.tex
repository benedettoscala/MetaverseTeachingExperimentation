\documentclass{article}
\usepackage{enumitem}
\usepackage{amsmath, amssymb}
\usepackage[a4paper,margin=1in]{geometry}

\begin{document}

\section*{Domande pre-test}

\begin{enumerate}[label=\textbf{Domanda \arabic*.}]

\item Qual è uno degli algoritmi più noti nel campo del calcolo quantistico?
\begin{enumerate}[label=\Alph*.]
    \item Algoritmo di Schrödinger
    \item Algoritmo di Shor
    \item Algoritmo di Turing
    \item Algoritmo di Huffman
\end{enumerate}
Risposta corretta: B

\item Come viene rappresentato un bit 0 in notazione vettoriale?
\begin{enumerate}[label=\Alph*.]
    \item \( \begin{pmatrix} 0 \\ 1 \end{pmatrix} \)
    \item \( \begin{pmatrix} 1 \\ 1 \end{pmatrix} \)
    \item \( \begin{pmatrix} 1 \\ 0 \end{pmatrix} \)
    \item \( \begin{pmatrix} 0 \\ 0 \end{pmatrix} \)
\end{enumerate}
Risposta corretta: C

\item Quale delle seguenti operazioni su un bit è reversibile?
\begin{enumerate}[label=\Alph*.]
    \item Negation
    \item Constant Zero
    \item Constant One
    \item Tutte le precedenti
\end{enumerate}
Risposte corrette: A

\item Nella notazione di Dirac, come viene rappresentato un vettore colonna \([1, 0]^T\)?
    \begin{enumerate}[label=\Alph*.]
        \item \(|0\rangle\)
        \item \(|1\rangle\)
        \item \(|2\rangle\)
        \item \(|3\rangle\)
    \end{enumerate}
    Risposta corretta: A

\item Cosa rappresenta il ``limite di von Neumann-Landauer''?
\begin{enumerate}[label=\Alph*.]
    \item La massima quantità di energia che un computer può utilizzare
    \item La minima quantità di energia necessaria per un calcolo che cancella informazioni
    \item La velocità massima di un computer quantistico
    \item La quantità di qubit in un computer quantistico
\end{enumerate}
Risposta corretta: B

\item Che cos'è il prodotto tensore tra due vettori?
\begin{enumerate}[label=\Alph*.]
    \item La somma dei vettori
    \item Il prodotto scalare dei vettori
    \item Un nuovo vettore ottenuto moltiplicando ogni elemento del primo vettore con ogni elemento del secondo
    \item La differenza tra i vettori
\end{enumerate}
Risposta corretta: C

\item Cosa fa l'operazione CNOT (Conditional NOT)?
\begin{enumerate}[label=\Alph*.]
    \item Inverte sempre il bit di controllo
    \item Inverte il bit target se il bit di controllo è 1
    \item Inverte il bit target se il bit di controllo è 0
    \item Inverte sempre entrambi i bit
\end{enumerate}
Risposta corretta: B

\item Qual è la differenza fondamentale tra un qubit e un bit classico?
\begin{enumerate}[label=\Alph*.]
\item Un qubit può rappresentare più di due stati alla volta
\item Un qubit può essere in una sovrapposizione di stati 0 e 1
\item Un qubit può memorizzare più informazioni di un bit classico quando misurato
\item Un qubit è più veloce di un bit classico nell'eseguire calcoli
\end{enumerate}
Risposta corretta: B

\item Quale delle seguenti porte è considerata la porta quantistica universale, cioè una porta da cui possono essere costruite tutte le altre porte?
\begin{enumerate}[label=\Alph*.]
    \item Porta Pauli-X
    \item Porta Hadamard
    \item Porta Toffoli
    \item Porta C-NOT
\end{enumerate}
Risposta corretta: C

\item Quanti "uno" possono essere presenti nel vettore dopo la misurazione di più qubit in sovrapposizione?
\begin{enumerate}[label=\Alph*.]
    \item Uno
    \item Due
    \item Tre
    \item Quattro
\end{enumerate}
Risposta corretta: A

\item È possibile eseguire calcoli classici su un computer quantistico?
\begin{enumerate}[label=\Alph*.]

\item No, un computer quantistico può solo eseguire algoritmi quantistici
\item Sì, ma solo per algoritmi che possono essere quantizzati
\item Sì, un computer quantistico può simulare un computer classico
\item No, poiché i computer quantistici utilizzano qubit invece di bit
\end{enumerate}
Risposta corretta: C

\item Cosa succede se applichiamo la porta Hadamard a un qubit in sovrapposizione?
\begin{enumerate}[label=\Alph*.]
    
    \item Il qubit viene distrutto
    \item Il qubit rimane in sovrapposizione
    \item Il qubit viene misurato
    \item Il qubit torna a uno stato classico
\end{enumerate}
Risposta corretta: D

\item Cosa rappresenta \(|1 0\rangle\) nel quantum computing?
\begin{enumerate}[label=\Alph*.]
    \item Un singolo qubit nello stato \(|1\rangle\)
    \item Un sistema di due qubit, dove il primo è nello stato \(|1\rangle\) e il secondo nello stato \(|0\rangle\)
    \item Una sovrapposizione dei stati \(|1\rangle\) e \(|0\rangle\)
    \item Un sistema di due qubit entangled in uno stato particolare
\end{enumerate}
Risposta corretta: B

\item Quale delle seguenti affermazioni descrive correttamente l'entanglement quantistico?
\begin{enumerate}[label=\Alph*.]
    \item Due qubit entangled condividono informazioni tra di loro, ma possono essere separati spazialmente.
    \item L'entanglement significa che due qubit sono nello stesso stato fisico.
    \item L'entanglement si verifica quando due qubit hanno la stessa frequenza di oscillazione.
    \item Due qubit entangled devono sempre essere fisicamente vicini tra loro.
\end{enumerate}
Risposta corretta: A


\item Chi è che ha raggiunto la "quantum supremacy" nel 2019?
\begin{enumerate}[label=\Alph*.]
    \item Microsoft
    \item Google
    \item Apple
    \item IBM
\end{enumerate}
Risposta corretta: B

\end{enumerate}
\end{document}

